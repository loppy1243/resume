\projectheader
   {}%
   {\normalfont\underline{\raisebox{0pt}[\height][0.4ex]{GitHub}}}%
   {\normalfont\underline{\raisebox{0pt}[\height][0.4ex]{Tech}}}%
   {}
\vspace{1ex}

\projectheader{This Resume}{resume}\LaTeX{2022}
\vspace{\dimexpr\bodyspace-\headerspace}

\begin{project}{Geometric/Clifford Algebra Library}{GeomAlg}{Julia}{In progress}
    \item Uses \textttbf{Cassette.jl} to treat quadratic forms
        \emph{as an execution context} for multivectors
        rather than associated data.
    \item \emph{Extends} the basic multivector type
        described in Breuils, Nozick, and Fuchs'
        \textit{A Geometric Algebra Implementation using Binary Tree} (2017)
        to work with \emph{all quadratic forms}
        rather than \emph{just diagonal} ones.
\end{project}

\begin{project}{Personal Website}{fract-refract}{Julia, JavaScript, CSS, HTML}{2022}
    \item Built with the \textttbf{Franklin.jl} static site generator for Julia,
        intended to host a blog and any of my many other writings.
    \item Footnotes written inline are displayed on the side
        and spaced without overlap.
    Adjusts correctly when the page is \emph{resized}.
\end{project}

\begin{project}{Ruled Paper Printer}{}{PostScript, Bash}{2020}
    \item Generates \emph{PostScript} files
        that can be \emph{sent to a printer} to print lines and markings
        for \emph{practicing calligraphy}.
\end{project}

\begin{project}{Exercises in Parallel Programming}{cmse401-git}%
               {C, Julia, CUDA, OpenMP, MPI}%
               {2019}
    \item Implemented parallel methods in topics like
        \emph{solving ODEs}, 
        \emph{matrix operations},
        \emph{image processing},
        and \emph{actor models}.
    \item Ran all programs
        on MSU's \emph{High Performance Computing Cluster},
        scheduled with the \emph{SLURM} workload manager.
\end{project}

\begin{project}{Ising Model Restricted Boltzmann Machine}{IsingBoltzmann}{Julia, CUDA}{2019}
    \item Based on Torlai and Melko's
        \textit{Learning Thermodynamics with Boltzmann Machines} (2016).
    \item Learns from \emph{Metropolis-Hastings} samples of an Ising model;
        converges in \emph{${\sim}$250 epochs}
        with \emph{KL-divergence ${\sim}$0.3\,bits}.
    \item Implements optional \emph{CUDA} acceleration
        via the \textttbf{CuArrays.jl} and \textttbf{CUDAnative.jl} packages.
\end{project}

\begin{project}{Simple x86 Bootloader}{bootloader}{x86, NASM, BIOS}{${\sim}$2012}
    \item Two-stage bootloader written in \emph{Intel syntax x86 assembly}
        for the \emph{NASM} assembler.
    \item First stage goes in the first \emph{512-byte} hard disk sector;
        it loads the second stage,
        printing ``Hello, World!'' using the \emph{BIOS}.
\end{project}

\begin{project}{Music Notation Compiler}{music}{Racket, x86-64}{${\sim}$2012}
    \item Supports musical notes, tempos, key signatures, and octaves;
        and \emph{loops}, \emph{conditionals},
        and \emph{functions with parameters}.
    \item General structure:
        Lexer $\to$ Parser $\to$ AST $\to$ IR $\to$ back-end.
    Parser made using \emph{generator similar to YACC}.
    \item Back-ends are an \emph{interpreter} written in Racket
        and a \emph{native binary} generator.
\end{project}
