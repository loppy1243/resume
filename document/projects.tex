\projectheader
   {}%
   {\normalfont\underline{\raisebox{0pt}[\height][0.4ex]{GitHub}}}%
   {\normalfont\underline{\raisebox{0pt}[\height][0.4ex]{Tech}}}%
   {}
\vspace{1ex}

\begin{project}{This Resume}{resume}\LaTeX{2022}
    \item You're looking at branch \texttt{master}.
\end{project}

\begin{project}{Geometric/Clifford Algebra Library}{GeomAlg}{Julia}{In progress}
    \item Uses the \textttbf{Cassette.jl} package
        for the ergonomic and performant handling of quadratic forms
        \textit{as an execution context} for multivectors
        rather than an additional piece of data to be passed around.
    \item Implements a basic multivector type
        described in Breuils, Nozick, and Fuchs'
        \textit{A Geometric Algebra Implementation using Binary Tree} (2017)
        and extends it to work with a kind of tridiagonal quadratic form
        that all others are equivalent to.
    \item Plan to implement De Keninck and Dorst's
        \textit{Hyperwedge} (2020) algorithm
        for efficiently computing outer products of vectors.
\end{project}

\begin{project}{Personal Website}{fract-refract}{Julia,JS,CSS,HTML}{2022}
    \item Built with the \textttbf{Franklin.jl} static site generator for Julia,
        intended to host a blog and any of my many other writings.
    \item Uses \emph{JavaScript} and \emph{CSS}
        to display side notes to the side of the main text,
        placed on the page with an animation,
        spacing themselves out so they don't overlap,
        and displaying correctly when the page is resized.
\end{project}

\begin{project}{Ruled Paper Printer}{}{PostScript, Bash}{2020}
    \item Customizably generates a \emph{PostScript} file
        that can be sent to a printer to print lines and markings
        for practicing calligraphy.
\end{project}

\begin{project}{Ising Model RBM}{IsingBoltzmann}{Julia, CUDA}{2019}
    \item Implements a \emph{restricted Boltzmann machine} (RBM)
        that learns an Ising model following Torlai and Melko's
        \textit{Learning Thermodynamics with Boltzmann Machines} (2016).
    \item Estimates the Kullback-Leibler divergence
        (the loss function for the RBM)
        using the contrastive divergence algorithm.
    \item Tests the validity of the RBM
        by comparing it to \emph{Metropolis-Hastings} samples
        of the Ising model.
    \item Implements optional \emph{CUDA} acceleration
        via the \textttbf{CuArrays.jl} and \textttbf{CUDAnative.jl} packages.
\end{project}

\begin{project}{Exercises in HPC}{cmse401-git}%
               {\parbox[t]{2in}%
                   {C, CUDA, OpenMP, MPI,\\
                    Julia, Bash, AWK, gnuplot,\\
                    Python, Makefiles, SLURM}}%
               {2019}
    \item Implementation of HPC methods in topics such as
        solving ODEs, 
        matrix operations,
        image processing,
        and actor models.
    \item Detailed numerical comparisons with na\"ive methods,
        with writeups on GitHub in each subfolder.
    \item All programs were run
        on Michigan State University's High Performance Computing Cluster,
        which runs \emph{Linux} and is accessed via \emph{SSH}.
    Programs were scheduled with the \emph{SLURM} workload manager.
\end{project}

\begin{project}{Exercises in Computational Physics}{PHY480}{C++, GSL, gnuplot}{2019}
    \item Covers topics such as
        summation order,
        integration/differentiation methods,
        and truncated Hamiltonian diagonalization.
    \item Detailed numerical comparisons of errors and performance.
    \item Extensive use of \emph{Makefiles} for building all associated items
        and \emph{gnuplot} for analyzing generated data.
    \item Occasional use of \emph{OpenMP} to speed up stupidly parrallel loops.
\end{project}

\begin{project}{Electron Incidence Inference}{BetaML\_NN}{Julia}{2018}
    \item Predicts the location of an electron
        incident on a rectangular array of cells
        by first predicting which cell with a
        \emph{convolutional neural network},
        and then predicting a more precise location within that cell
        with a \emph{dense} neural network.
    \item The model performs \emph{significantly better}
        than a model that selects the highest energy cell
        and a random point within it.
    \item Reads in, parses, and normalizes CSV data.
    \item Uses the \textttbf{Flux.jl} \emph{machine learning} package.
\end{project}

\begin{project}{Simple x86 Bootloader}{bootloader}{x86}{${\sim}$2012}
    \item Two-stage bootloader written in \emph{Intel syntax x86 assembly}
        for the \emph{NASM} assembler.
    \item First stage fits within 512\,kB.
    Put at the start of the first sector of an MBR hard disk,
        it loads a second stage at the start of the second sector.
    The second stage prints ``Hello, World!'' using the \emph{BIOS}.
\end{project}

\begin{project}{Music Notation Compiler}{music}{Racket, x86-64}{${\sim}$2012}
    \item Compiles a text file of a custom music \emph{DSL} into beeps.
    \item Supports musical notes, tempos, key signatures, and octaves;
        and \emph{loops}, \emph{conditionals},
        and \emph{functions with parameters}.
    \item Uses a \emph{parser-generator}
        (part of the Racket standard library)
        similar to YACC to parse the source into an AST;
        this is transformed into an assembly-like IR;
        and this is passed to a back-end.
    \item Included back-ends are:
        an \emph{interpreter} that interfaces with an external beep program;
        and a \emph{native binary} generator
        that performs floating-point calculations
        in \emph{x86-64 assembly}
        and interfaces with the \emph{C standard library}.
\end{project}
